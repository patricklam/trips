\documentclass[11pt]{article}
\begin{document}
\thispagestyle{empty}

\begin{center}
\begin{large}
\textbf{Patrick Lam} \\
\textbf{Trip Report: Mounts Cabot and Waumbek} \\
\textbf{July 8, 2007}
\end{large}
\end{center}

After hiking the other 46 New Hampshire 4000-footers (we had just done
Garfield, Galehead, S. Twin, and Zealand the previous day), David
Wentzlaff and I made a plan to hike Mounts Cabot and Waumbek as a
north-to-south traverse. We would spot my car at the end of the Starr King
Trail and hike the Unknown Pond trail to the Kilkenny Ridge trail,
through the Bulge and the Horn, up to Mount Cabot, then to the three
Weeks, over Waumbek, and back to my car. Based on our past experience,
we estimated that this 20.1 mile hike would take us a bit over 10 hours:
our constant hiking rate is about 2 mph, independent of elevation gain.

Sunday morning was rainy, cold and windy. We started our hike at
8:45AM with a temperature of 15C. Soon enough, we reached Unknown
Pond, described as a remote jewel of the White Mountains. Remote, yes;
nice, yes; a jewel, not quite. Perhaps different weather would have
increased its appeal.

We then reached the Kilkenney Ridge trail and took the detour to the
Horn.  I noticed a couple of rock anchors in the Horn; it looks like
it would be fun to boulder, but I did not bring any suitable shoes for
climbing, only sandals, and the weather was still not particularly
nice. As was often the case in our New Hampshire adventures, we did
not benefit from any views, despite the promises of the guidebooks. Or
rather, we got another panoramic view of the inside of a cloud.

Upon reaching the summit of Mount Cabot, we took a relatively brief
lunch break inside Cabot Cabin, and managed to shake off a bit of the
water that was accumulating on us. Our lunch of peanut butter and
jelly sandwiches was as satisfying as usual and fuelled us for the 14
remaining miles. We considered changing our plan so that we would instead
hike up Cabot and back and up Waumbek and back; however, such a plan
does not save much distance, only elevation, so I preferred the traverse.

Continuing along, we reached Terrace (but did not take the 0.1mi spur
to the summit of Terrace) and the three Weeks, which aren't really
that related; each of these mountains is not lacking in prominence.
Also, by this point, it had stopped raining and warmed up quite a bit.
There also were no more views except of the blue sky.

After South Weeks, we knew that it would be difficult to avoid
finishing these mountains. We climbed Mount Waumbek and enjoyed
the ruins on Mount Starr King, then hiked the Starr King trail,
finishing at 7:30PM.

\end{document}
